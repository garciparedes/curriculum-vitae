%%%%%%%%%%%%%%%%%%%%%%%%%%%%%%%%%%%%%%%%%
% Friggeri Resume/CV
% XeLaTeX Template
% Version 1.2 (3/5/15)
%
% This template has been downloaded from:
% http://www.LaTeXTemplates.com
%
% Original author:
% Adrien Friggeri (adrien@friggeri.net)
% https://github.com/afriggeri/CV
%
% License:
% CC BY-NC-SA 3.0 (http://creativecommons.org/licenses/by-nc-sa/3.0/)
%
% Important notes:
% This template needs to be compiled with XeLaTeX and the bibliography, if used,
% needs to be compiled with biber rather than bibtex.
%
%%%%%%%%%%%%%%%%%%%%%%%%%%%%%%%%%%%%%%%%%

\documentclass[]{friggeri-cv} % Add 'print' as an option into the square bracket to remove colors from this template for printing
\usepackage{verbatim}
\usepackage{hyperref}

\addbibresource{bibliography.bib} % Specify the bibliography file to include publications


\begin{document}

\header{Sergio}{GarcíaPrado}{junior computing scientist} % Your name and current job title/field

%----------------------------------------------------------------------------------------
%	SIDEBAR SECTION
%----------------------------------------------------------------------------------------

\begin{aside} % In the aside, each new line forces a line break
\section{contact}
7 3º C, La Paz
34004 Palencia
Spain
~
+34 696 904 878
~
\href{mailto:garciparedes@gmail.com}{garciparedes@gmail.com}
\href{http://garciparedes.me}{garciparedes.me}
\href{http://facebook.com/garciparedes}{fb://garciparedes}
\section{languages}
spanish mother tongue
english fluency
\section{programming}
{\color{red} $\varheartsuit$}  Java, Python
JavaScript, C, C++
CSS3 \& HTML5
\end{aside}

%----------------------------------------------------------------------------------------
%	SUMMARY SECTION
%----------------------------------------------------------------------------------------

\section{summary}


%------------------------------------------------

I'm a student of Computer Engineering with a great desire to learn. I especially like the Computing branch, ie, data structures, algorithms, and everything that has to do with scheduling things. Although I do not close the door to anything. 

Currently I've started a personal project (EvaluaMe) with which I am putting into practice the knowledge acquired at university.

%------------------------------------------------


%----------------------------------------------------------------------------------------
%	EDUCATION SECTION
%----------------------------------------------------------------------------------------

\section{education}

\begin{entrylist}

%------------------------------------------------

\entry
{2013--Now}
{Degree {\normalfont of Computer Engineering} }
{University of Valladolid, Spain}
{Specialization in Computing}


%------------------------------------------------

\end{entrylist}


%----------------------------------------------------------------------------------------
%	AWARDS SECTION
%----------------------------------------------------------------------------------------

\section{awards}

\begin{entrylist}

%------------------------------------------------

\entry
{2016}
{Codes and Cryptography}
{University of Valladolid, Spain}
{Score: 10.0/10.0 with Honors.}

\entry
{}
{Algorithms and Computing}
{University of Valladolid, Spain}
{Score: 9.5/10.0 with Honors.} 

\\
\entry
{2015}
{Operating Systems Structures}
{University of Valladolid, Spain}
{Score: 9.5/10.0 with Honors.}

%------------------------------------------------

\end{entrylist}

%----------------------------------------------------------------------------------------
%	SKILLS SECTION
%----------------------------------------------------------------------------------------

\section{skills}



\newpage

%----------------------------------------------------------------------------------------
%	APPLICATIONS SECTION
%----------------------------------------------------------------------------------------
\section{applications}

\begin{entrylist}
  
  \entry
    {2016}
    {Calculator with arbitrary-precision integer arithmetic. }
    {University of Valladolid, Spain \\ 
    \href{https://github.com/ismtabo/cc_bigintegers}{\textit{github.com/ismtabo/cc\_bigintegers}}}
    {Classwork of Codes and Cryptography. The project involves the implementation in Java of a Calculator with  arbitrary integer arithmetic precision.}
    \\
  
  \entry
    {2015}
    {Implementation of a Random Text Generator}
    {University of Valladolid, Spain \\
    \href{https://github.com/garciparedes/Generador-de-texto-aleatorio}{\textit{github.com/garciparedes/Generador-de-texto-aleatorio}} }
    {ClassWork of Data Structures and Algorithms. The project involves the implementation of a random text generator implemented in Java.}
  
  \entry
    {}
    {EvaluaMe}
    {EvaluaMe \\
    \href{https://play.google.com/store/apps/details?id=com.garciparedes.evaluame}{\textit{Play Store: EvaluaMe}} }
    {Experiment aimed to measure viral spreading of content across the blogosphere.}
    
    \entry
    {}
    {Modeling a Burguer Shop.}
    {University of Valladolid, Spain \\ 
    \href{https://github.com/garciparedes/InfoBurguer}{\textit{github.com/garciparedes/InfoBurguer}}}
    {Classwork at Object Oriented Programming. The project consists in modeling an Burguer Shop  in Java.}

\end{entrylist}

%----------------------------------------------------------------------------------------
%	PUBLICATIONS SECTION
%----------------------------------------------------------------------------------------

\section{publications}

\begin{entrylist}
  
  \entry
    {2016}
    {NP-Complete problem of Job Shop Scheduling. }
    {University of Valladolid, Spain \\ 
    \href{https://github.com/garciparedes/Job-Shop-Scheduling-NP-Complete}{\textit{github.com/garciparedes/Job-Shop-Scheduling-NP-Complete}}}
    {Classwork of Codes and Cryptography. The project involves the implementation in Java of a Calculator with  arbitrary integer arithmetic precision.}
  
  \entry
    {}
    {Longest Common Subsequence.}
    {University of Valladolid, Spain \\
    \href{https://github.com/garciparedes/Longest-Common-Subsequence}{\textit{github.com/garciparedes/Longest-Common-Subsequence}} }
    {ClassWork of Data Structures and Algorithms. The project involves the implementation of a random text generator implemented in Java.}
    
    \entry
    {}
    {Closest pair of points in n-dimensional space.}
    {University of Valladolid, Spain \\ 
    \href{https://github.com/garciparedes/Closest-Pair-of-Points}{\textit{github.com/garciparedes/Closest-Pair-of-Points}}}
    {Classwork at Object Oriented Programming. The project consists in modeling an Burguer Shop  in Java.}

\end{entrylist}
%----------------------------------------------------------------------------------------
%	INTERESTS SECTION
%----------------------------------------------------------------------------------------

\section{interests}

\textbf{professional:} computing problems, data analysis, algorithms, machine learning, design patterns, web design, software design, internet of things \\
\textbf{personal:} motor sports, rap and classic music, cooking, technology, social philosophy



\end{document}